\section{Person Isolation}
\label{design:person isolation}
This section details two possible methods of person isolation.
The first method uses the Kinect colour stream whereas the second uses the depth stream.
Both methods make use of the Kinect's ability to isolate a skeleton associated with a person.
The use of this skeleton meant that the group did not have to make use of computationally heavy computer vision algorithms to isolate a person \todo{example}.\\

\subsection{Colour Based Isolation}
The Kinect can determine whether a colour pixel is associated with a detected skeleton. Hence if a pixel is not associated, the colour value for that pixel can be set to white.
It is expected 

\subsection{Depth Based Isolation


Instead, the method developed used the skeleton to determine the approximate depth of the person. 
Any point whose depth value is outside a delta of the skeleton's depth is discarded.\\ 

Cutting off base depth alone is not enough, as this method leaves a \textit{ring} of equal distant points in-line with the person. 
To eliminate this ring, the positions of left and right most point of the person (i.e. the HandLeft joint and the HandRight joint) are calculated and anything outside of this range is also discarded. \\

Section \ref{testing:person isolation} details the effectiveness of this approach.