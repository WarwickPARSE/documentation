\section{Person Isolation}
\label{design:person isolation}
This section details the creation of a method to isolate a person from the background data. An advantage of using a Kinect over another camera is that the Kinect is able to isolate a skeleton associated with a person. This skeleton meant that the group did not have to make use of computationally heavy computer vision algorithms to isolate a person \todo{example}.\\

Instead, the method developed used the skeleton to determine the approximate depth of the person (using the HipCenter joint depth). Any point whose depth value is outside a delta of the HipCenter's depth is discarded.\\ 

Cutting off base depth alone is not enough, as this method leaves a \textit{ring} of equal distant points in-line with the person. To eliminate this ring, the positions of left and right most point of the person (i.e. HandLeft and HandRight) are calculated and anything outside of this range is also discarded. \\

Section \ref{testing:person isolation} details the effectiveness of this approach.