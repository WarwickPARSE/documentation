\chapter{Research}

\label{research}

\section{Introduction}

The use of image processing and it's associated computational techniques for the analysis, enhancement, compression and reconstruction of images has been used in the medical domain since the late 1960's and early 1970's to develop an understanding of the processes and composition of the Human Body \cite{HansonHistory}. More recently, technological innovations in the area of determining body composition accurately have become more mature \cite{WeyersBodpod} and thus more relevant to the study of the human body as health conditions and the research of disease behaviour has been identified as being dependent on the body's internal composition. Techniques that use MRI scanning or other radiation based techniques offer significant insight into the composition of the body's internal structure. Particular studies have utilised whole body MRI scanning to produce images that can then be used to determine composition due to it's large coverage, low impact and ability to repeatedly acquire images \cite{KullbergMRI}. 

\cite{SteinkampComposition}. More recently, Image processing has also been used as a means of calibrating the scanning procedure for patients who require repeated ultrasound or other similar non-intrusive scans of a particular area of the body during a course of treatment or for the diagnosis or analysis of a course of therapy or weight loss programme. Previous research has provided the basis for markerless recognition by using points of reference on the human body for automated feature extraction /cite{LeongMarkerless}. The identification of the components of the human body without markers forms the basis for our research into the identification of sites for ultrasound scanning without the use of significant cues or markers.

Range Imaging, especially HDR is a relatively new technique being used in the context of image processing. Range Imaging produces 2D images with depth cues and while their relative applications in Medical Imaging have been small to date their potential is evident in previous applications outside of the context of medicine with pixel values corresponding to distance from the device already being used in the calibration of patient positioning for MRI scans using range imaging sensor devices.

This research section will consider the use of Body Composition in the analysis and diagnosis of medical state and conditions that a patient may be subject to. It will consider the current standards for the determination and analysis of Body Composition and what the future direction of image processing for the determination of body composition is. Markerless recognition in the context of scan registration and patient calibration will also be explored and technologies and practices identified where the standards are implemented. Range Imaging techniques and devices will also be introduced within the context of these previous two medical sub-domains in order to inform our research into related work and project design and implementation.

\section{Body Composition}

The analysis

\section{Markerless Recognition}

\section{Range Imaging}

\section{Related Work}

