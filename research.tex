\chapter{Research}

\label{research}

\section{Introduction}

The use of image processing and it's associated computational techniques for the analysis, enhancement, compression and reconstruction of images has been used in the medical domain since the late 1960's and early 1970's to develop an understanding of the processes and composition of the Human Body \cite{HansonHistory}. 
More recently, technological innovations in the area of determining body composition accurately have become more mature \cite{WeyersBodpod} and thus more relevant to the study of the human body as health conditions and the research of disease behaviour has been identified as being dependent on the body's internal composition \cite{SteinkampComposition}. 
Techniques that use MRI scanning or other radiation based techniques offer significant insight into the composition of the body's internal structure. 
Particular studies have utilised whole body MRI scanning to produce images that can then be used to determine composition due to it's large coverage, low impact and ability to repeatedly acquire images \cite{KullbergMRI}. 

Image processing has also been used as a means of calibrating the scanning procedure for patients who require repeated ultrasound or other similar non-intrusive scans of a particular area of the body during a course of treatment or for the diagnosis or analysis of a course of therapy or weight loss programme. 
Previous research has provided the basis for markerless recognition by using points of reference on the human body for automated feature extraction \cite{LeongMarkerless}. 
The identification of the components of the human body without markers forms the basis for our research into the identification of sites for ultrasound scanning without the use of significant cues or markers.

Range Imaging, especially HDR is a relatively new technique being used in the context of image processing. 
Range Imaging produces 2D images with depth cues and while their relative applications in Medical Imaging have been small to date their potential is evident in previous applications outside of the context of medicine with pixel values corresponding to distance from the device already being used in the calibration of patient positioning for MRI scans using range imaging sensor devices.

This research section will consider the use of Body Composition in the analysis and diagnosis of medical state and conditions that a patient may be subject to. 
It will consider the current standards for the determination and analysis of Body Composition and what the future direction of image processing for the determination of body composition is.
Markerless recognition in the context of scan registration and patient calibration will also be explored and technologies and practices identified where the standards are implemented. 
Range Imaging techniques and devices will also be introduced within the context of these previous two medical sub-domains in order to inform our research into related work and project design and implementation.

\section{Body Composition}

The composition of the human body as determined by the physical fitness of a person is the relative percentages of fat, muscle, bone and other tissues in the body. 
This ratio of muscle to fat tissue is typically responsible for the outward appearance of a person and determines their leanness with respect to their body fat percentage. 
Body mass and volume is most commonly determined using the Body Mass Index (BMI) and this measurement has been used in clinical trials for a number of years in the determination of health conditions and disease proliferation linked to a person's height and weight \cite{PrevalenceFlegal}. 
However, BMI is limited by the lack of information that can be gained from the correlation of height and weight in the determination of mass and is unable to detect excess body adiposity of people in the intermediate range of the BMI scale who may have reasonable weight to height ratios but excessive weight distribution or abdominal volume \cite{BMIAccuracy}.

\begin{figure}[t]
\label{bmi30}
	\centering
	\includegraphics[scale=0.6]{images/bmi30women.jpg}
	\caption{8 Women with a BMI of 30 but with different weight distribution}
\end{figure}

A new measure of body composition is the Body Volume Index (BVI) launched in 2010 as an improved anthropometric measurement for the discovery of weight information and distribution on the human body. 
BVI is ascertained using a scanner that operates using a 3D scanner to calculate measurements and risk factors that could be associated with a person's body shape and type through the capturing of their weight, volume and body fat \cite{BVIRelease}. 
The calculation of a person's Body Volume Index in cases of obese patients was identified as being linked  to bio markers of cardiovascular disease and highlights the benefits associated with methods of acquiring 3D images of the body in the motivation of measuring patient volume for weight loss.

\section{Background Subtraction and Person Isolation}
\label{background subtraction and person isolation}
In order to create a point cloud of a person, the tool kit must be able to isolate the person being scanned from the background environment. This is a well researched problem and is an issue for many applications in the computer vision field. Even though many background subtraction algorithms have been proposed in the literature, the problem of identifying moving objects in complex environment is still far from being completely solved \ref{Cheung2007}.\\

\subsection{Isolating Objects}
\label{isolating objects}

\subsubsection{General Approaches, Properties and Steps}
A common approach for isolating general objects is to perform background subtraction, which identifies moving objects from the portion of a video frame that differs significantly from a background model \ref{Cheung2007}.
This general approach allows the isolation of non humanoid objects such as cars, but is still worthy of note because any person isolation algorithm must shared some similarities with generic object isolation algorithms.\\

These similarities between the algorithms for object isolation and person isolation are that both must \ref{Cheung2007}:\begin{itemize}
  \item Be robust against changes in illumination
  \item Avoid detecting non-stationary background objects such as swinging leaves
  \item React quickly to changes
\end{itemize}

Approaches to object isolation vary from simple techniques such as frame differencing and adaptive median filtering, to more sophisticated probabilistic modelling techniques. 
While complicated techniques often produce superior performance, experiments \ref{Cheung2007} show that simple techniques such as adaptive median filtering can produce good results with much lower computational complexity.\\

In general, the four major steps in a background subtraction algorithm are preprocessing, background modelling, foreground detection, and data validation.
Preprocessing consists of a collection of simple image processing tasks that change the raw input video into a format that can be processed by subsequent steps and can involve noise reduction and frame size/rate reductions \ref{Cheung2007}\\.

Background modelling uses the new video frame to calculate and update a background model.
This background model provides a statistical description of the entire background scene and can be non-recursive or recursive. 
A non-recursive technique uses a sliding-window approach for background estimation. 
The technique stores a buffer of the previous $L$ video frames, and estimates the background image based on the temporal variation of each pixel within the buffer.
Non-recursive techniques are highly adaptive as they do not depend on the history beyond those frames stored in the buffer \ref{Cheung2007}, which may make the ideal for the tool kit.
On the other hand, the storage requirement can be significant if a large buffer
is needed to cope with slow-moving objects \ref{Cheung2007}, although that should not be a problem for the tool kit. Recursive techniques do not use a buffer but maintain a single background model based on each input frame. As a result, input frames from distant past could have an effect on the current background model \ref{Cheung2007} and hence may not be suitable for the tool kit.\\

Foreground detection then identifies pixels in the video frame that cannot be adequately explained by the background model and outputs them as a binary candidate foreground mask. 
Data validation examines the candidate mask and attempt to reduce false-positive or false-negative regions and eliminates those pixels that do not correspond to actual moving objects, and outputs the final foreground mask \ref{Cheung2007}.\\

\subsubsection{Frame Differencing, Median, Kalman and Gaussian Filtering.}
Chenug \ref{Cheung2007}
\section{Volume Estimation}
\label{design:volume estimation}
There are many techniques for estimating an arbitrary metric of a given 3 dimensional shape as well as the volume of the shape.
Focus will first be on the arbitrary metrics, such as height, to see if the can be extended to calculate volume, before moving to volume estimation methods.\\

\subsection{Estimating the Height of Trees}
LIDAR has previously been used to estimate the height of trees from above \cite{Maltamo2006}. Three laser scanning-based methods were used to compute the height of the trees, a direct prediction model for the stem volume at plot level, a volume prediction system based on the modelled percentiles of the basal area diameter distribution and a parameter prediction method used to determinate Weibull based basal area diameter distributions \cite{Frechet1927} for the plot-level stem volume prediction. The best results were obtained with the first method, i.e. the model that predicts plot-level stem volumes directly \cite{Maltamo2006}.\\

In order to calculate the height, the laser reflections from non-ground objects, such as trees and buildings were classified as non-ground hits using TerraScan software \cite{Solid2013}. Conversely other points were classified as ground hits.
Canopy height of a non-ground object is then calculated as the difference between the height of the non-ground object and the neighbouring ground points. 
The accuracy of this method was found to be better than $\pm$15cm \cite{Maltamo2006}, when compare to field-measured volumes obtained with the Finnish conventional method Inventory by Compartments \cite{Koivuniemi2006}.\\

It should be noted that in this method the LIDAR emitter was above the target \cite{Maltamo2006}, rather than in front of them as is likely in the case of the project. This method then translate to using the top and the bottom most point of the cloud to calculate height. However it may be able to be extended to calculate metrics such as depth and breath, from which a minimum bounding box could be determined.\\

A minimum bounding box is defined as follows in \ref{fig:bounding_box_definition}.\\

\begin{figure}[h]
\textit{Definition: For a point set in N dimensions, it refers to the box with the smallest measure (area, volume, or hypervolume in higher dimensions) within which all the points lie} \cite{Barequet2001}.
\caption {Minimum bounding box definition}
\label{fig:bounding_box_definition}
\end{figure}

For the purposes of the project, this box would be three dimensional and the volume of the minimum bounding box could be calculated as in Figure \ref{fig:calculating_the_volume_of_the_minimum_bounding_box}.\\

\begin{figure}[h]
\begin{center}
$Volume = (xmax -xmin) * (ymax - ymin) * (zmax - zmin)$
\end{center}
\caption{Calculating the volume of the minimum bounding box}
\label{fig:calculating_the_volume_of_the_minimum_bounding_box}
\end{figure}

\subsection{Determination of prostate volume by transrectal ultrasound}
The volumes of prostates has been calculated from ultrasound scanners previously using many methods. 
Many methods have been tested, including step-section planimetry and the elliptical volume method. After the volume of the prostate was estimated, all patients underwent subsequent radical prostatectomy or cystoprostatectomy and prostate specimen weights were compared with the results of each volume estimation method \cite{K1991}.\\ 

\subsubsection{Step-Section Planimetry}
Step-section planimetry (SSP) is so called because of the planimeter, a measuring instrument used to determine the area of an arbitrary two-dimensional shape by traversing the perimeter \cite{Bryant2011}. 
The method calculates the area at each step, uses these areas to calculate the volume of a step and then sums these volume to determine the total volume.\\

SSP is often assumed to be the most accurate means of volume measurement and in when used to calculate prostate volume \cite{K1991} exhibited a Pearson correlation coefficient of 0.93. 
Such a correlation coefficient indicates the method is highly accurate.
As the method only needs to traverse the perimeter once it will have a low running time which is essential as it will be called on many planes. As the method is traversing up the point cloud representation of an object, the method will run in $O(n)$, where n is the number of points in the point cloud. Also, whilst the area/volume is being calculated other useful metrics could be determined, such as perimeter of a step \footnote{or a plane}.\\

Similar method to SSP have been used in other areas of medicine, such as measuring brain, fetal lung and heart volumes \cite{Rosen1990,Rypens2001,Graham1971}.\\

\subsubsection{Elliptical volume}
Much like a bounding box, an ellipsoid can be formed around a person using their height, depth and breadth and the volume of this ellipsoid is then calculated.
For prostates, the elliptical volume method, demonstrated a correlation coefficient of 0.90 \cite{K1991}. Again, this suggests a high accuracy.
However, this high performance may be due to the roughly walnut shaped nature of a prostate \cite{D2003}. 
On a less spherical object, such as a person, the elliptical method may output a higher volume than the true value, as with the bounding box.\\

If the min and max values of a object were stored in it's point cloud, the ellipsoid could be computed in $O(1)$, similar to the bounding box. 
If the min and max values need to be computed on the fly, this would bring the complexity up to $O(n)$, where n is the number of points in the point cloud, putting its running time in the same order as the step-section method.
As elliptical volume is less accurate, takes the same time and cannot give other information, such as perimeter, step-section will be the bases of volume calculation in the toolkit.

\subsubsection{Convex Hulls}
Because the Kinect data is quite noise, it may be necessary to compute the convex hull of a plane to give more accurate data from the SSP method. Three convex hull methods were researched, Gift-Wrapping \cite{Cormen2001}, Quick Hull \cite{Barber1996} and the Kirkpatrick–Seidel algorithm.\\

Gift-Wrapping (GW), also known as Jarvis' march \cite{Jarvis1973}, is similar in two dimensions to the process of winding a string around the set of points. 
GW runs in $O(nh)$  \cite{Cormen2001} where $n$ is the number of points in the input and $h$ is the number of points in the hull. In the project case, $h$ is likely to be in the order of $O(n)$ as the points are expected to already be almost hull like. 
Hence GW can be, for the purposes of the project, said to run in $O(n^2)$. However, GW is simple to implement so may be the method implemented.\\

Quick Hull (QW) uses a divide and conquer approach similar to that of QuickSort, which its name derives from. Its average case complexity is considered to be O(n * log(n)), whereas in the worst case it takes O(n2) (quadratic).




%http://en.wikipedia.org/wiki/Gift_wrapping_algorithm
%http://en.wikipedia.org/wiki/QuickHull
%http://en.wikipedia.org/wiki/Kirkpatrick%E2%80%93Seidel_algorithm
%\subsection{Current Methods}

%\section{Markerless Recognition}

%\section{Range Imaging}

%\subsection{Traditional Imaging Equipment}

%\subsection{HDR Cameras}

%\subsection{IR Sensors}

%\subsubsection{Microsoft Kinect}

%\section{Related Work}