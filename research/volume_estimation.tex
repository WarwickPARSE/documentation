\section{Volume Estimation}
\label{design:volume estimation}
There are many techniques for estimate the volume of an arbitrary 3 dimensional shape. \\

LIDAR has previously been used to estimate the height of trees from above \cite{Maltamo2006}.\\
 
Three \todo{of the} laser scanning-based methods were compared were a direct prediction model for the stem volume at plot level, a volume
prediction system based on the modelled percentiles of the basal area diameter distribution and a parameter prediction method used to determinate Weibull-based basal area diameter distributions for the plot-level stem volume prediction. \\

The predicted volumes were also compared with field-measured volumes obtained with the Finnish conventional inventory by compartments. 
The best results were obtained with the first method, i.e. the model that predicts plot-level stem volumes directly \cite{Maltamo2006}.\\

In order to calculate the height, the laser reflections from non-ground objects, such as trees and buildings were classified as non-ground hits using TerraScan software \cite{Solid2013}. Conversely other points were classified as ground hits.
Canopy height of a non-ground object is then calculated as the difference between the height of the non-ground object and the neighbouring ground points. 
The accuracy of this method was found to be better than $\pm$15cm\cite{Maltamo2006}.\\