\section{Volume Estimation}
\label{design:volume estimation}
There are many techniques for estimating an arbitrary metric of a given 3 dimensional shape as well as the volume of the shape.
Focus will first be on the arbitrary metrics, such as height, to see if the can be extended to calculate volume.\\

LIDAR has previously been used to estimate the height of trees from above \cite{Maltamo2006}. Three laser scanning-based methods were used to compute the height of the trees, a direct prediction model for the stem volume at plot level, a volume prediction system based on the modelled percentiles of the basal area diameter distribution and a parameter prediction method used to determinate Weibull based basal area diameter distributions \cite{Frechet1927} for the plot-level stem volume prediction. The best results were obtained with the first method, i.e. the model that predicts plot-level stem volumes directly \cite{Maltamo2006}.\\

In order to calculate the height, the laser reflections from non-ground objects, such as trees and buildings were classified as non-ground hits using TerraScan software \cite{Solid2013}. Conversely other points were classified as ground hits.
Canopy height of a non-ground object is then calculated as the difference between the height of the non-ground object and the neighbouring ground points. 
The accuracy of this method was found to be better than $\pm$15cm, when compare to field-measured volumes obtained with the Finnish conventional inventory by compartments. \cite{Maltamo2006}.\\