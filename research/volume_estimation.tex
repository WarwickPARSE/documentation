\section{Volume Estimation}
\label{design:volume estimation}
There are many techniques for estimating an arbitrary metric of a given 3 dimensional shape as well as the volume of the shape.
Focus will first be on the arbitrary metrics, such as height, to see if the can be extended to calculate volume, before moving to volume estimation methods.\\

\subsection{Estimating the Height of Trees}
LIDAR has previously been used to estimate the height of trees from above \cite{Maltamo2006}. Three laser scanning-based methods were used to compute the height of the trees, a direct prediction model for the stem volume at plot level, a volume prediction system based on the modelled percentiles of the basal area diameter distribution and a parameter prediction method used to determinate Weibull based basal area diameter distributions \cite{Frechet1927} for the plot-level stem volume prediction. The best results were obtained with the first method, i.e. the model that predicts plot-level stem volumes directly \cite{Maltamo2006}.\\

In order to calculate the height, the laser reflections from non-ground objects, such as trees and buildings were classified as non-ground hits using TerraScan software \cite{Solid2013}. Conversely other points were classified as ground hits.
Canopy height of a non-ground object is then calculated as the difference between the height of the non-ground object and the neighbouring ground points. 
The accuracy of this method was found to be better than $\pm$15cm \cite{Maltamo2006}, when compare to field-measured volumes obtained with the Finnish conventional method Inventory by Compartments \cite{Koivuniemi2006}.\\

It should be noted that in this method the LIDAR emitter was above the target \cite{Maltamo2006}, rather than in front of them as is likely in the case of the project. This method then translate to using the top and the bottom most point of the cloud to calculate height. However it may be able to be extended to calculate metrics such as depth and breath, from which a minimum bounding box could be determined.\\

A minimum bounding box is defined as follows in \ref{fig:bounding_box_definition}.\\

\begin{figure}[h]
\textit{Definition: For a point set in N dimensions, it refers to the box with the smallest measure (area, volume, or hypervolume in higher dimensions) within which all the points lie} \cite{Barequet2001}.
\caption {Minimum bounding box definition}
\label{fig:bounding_box_definition}
\end{figure}

For the purposes of the project, this box would be three dimensional and the volume of the minimum bounding box could be calculated as in Figure \ref{fig:calculating_the_volume_of_the_minimum_bounding_box}.\\

\begin{figure}[h]
\begin{center}
$Volume = (xmax -xmin) * (ymax - ymin) * (zmax - zmin)$
\end{center}
\caption{Calculating the volume of the minimum bounding box}
\label{fig:calculating_the_volume_of_the_minimum_bounding_box}
\end{figure}

\subsection{Determination of prostate volume by transrectal ultrasound}

The volumes of prostates has been calculated from ultrasound scanners previously using many methods \cite{K1991}. 
Many methods have been tested, including step-section planimetry and the elliptical volume method.\\ 

All patients underwent subsequent radical prostatectomy or cystoprostatectomy. Prostate specimen weights were compared with the results of each volume estimation method. 

\subsubsection{Step-Section Planimetry}
Step-section planimetry is so called because of the planimeter, a measuring instrument used to determine the area of an arbitrary two-dimensional shape by traversing the perimeter \cite{Bryant2011}. 
The method calculates the area at each step, uses these areas to calculate the volume of a step and then sums these volume to determine the total volume.\\

Step-section planimetry is often assumed to be the most accurate means of volume measurement and in when used to calculate prostate volume \cite{K1991} exhibited a Pearson correlation coefficient of 0.93. 
Such a correlation coefficient indicates the method is highly accurate.
As the method only needs to traverse the perimeter once it will have a low running time which is essential as it will be called on many planes. Also, whilst the area/volume is being calculated other useful metrics could be determined, such as perimeter of a step \footnote{or a plane}.\\

\subsubsection{Elliptical volume}
Much like a bounding box, an ellipsoid can be formed around a person using their height, depth and breadth and the volume of this ellipsoid is then calculated. For prostates, the elliptical volume method, demonstrated a correlation coefficient of 0.90 \cite{K1991}. Again, this suggests a high accuracy. However, this high performance may be due to the roughly walnut shaped nature of a prostate \todo{cite}. On a less spherical object, such as a person, the elliptical method may output a higher volume than the true value. If the min and max values of a object were stored in it's point cloud, the ellipsoid could be computed in $O(1)$, similar to the bounding box. If the min and max values need to be computed on the fly, this would up the complexity to $O(n)$, where n is the number of points in the point cloud, putting its running time in the same order as the step-section method. As elliptical volume is less accurate, takes the same time and cannot give other information, such as perimeter, step-section will be the bases of volume calculation in the tookit.

although elliptical volume did not perform as highly as step-section planimetry.

The most accurate method to estimate prostate weight (r = 0.94) was a variation of the prolate spheroid formula, expressed as pi/6 (transverse dimension)2 (anteroposterior dimension). When different volume ranges were considered, this prolate spheroid formula provided the closest estimate of weight in glands of less than 40 gm. and those in the 40 to 80 gm. range. The most accurate method to estimate prostates weighing greater than 80 gm. was the formula pi/6 (transverse dimension)3. 