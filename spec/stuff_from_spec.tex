\section{Problem Statement}
\label{spec: problem statement}
Computer Vision and object recognition are becoming more prevalent in day-to-day life, entering the home through commodity hardware such as the Microsoft Kinect. Because of this, many fields are looking to use the Kinect to automate or improve many of their processes. One such field is medicine, where medical imaging, specifically range imaging is rapidly developing in a number of different contexts, such as patient setup, and being applied in a variety of situations. Advancements in this area are leading to better patient monitoring, diagnosis and treatment of patients in various sub domains in Medicine.\\  

The PARSE project aims to create an application toolkit to increase the reliability and accuracy of the measurements taken when monitoring weight loss and body size variations of a patient. The toolkit will provide an easy to use interface allowing for medical personnel to take measurements of body volume, limb circumferences and relative positions of ultrasound scanners on the surface of the body. These measurements can be stored persistently and be recalled later as part of a patient’s medical record. The remit for achieving this will be in our use of commodity hardware - specifically the Microsoft Kinect.\\ 

\section{Motivation}
\label{spec:motivation}
The motivation for this project came from several limitations associated with cost and the accessibility of current practice for measuring weight loss, body shape and the calibration/configuration of non-invasive medical scanning equipment. Body volume is typically measured using multiple sensors \cite{Bauer2011} or expensive air displacement plethysmography equipment \cite{Izadi2011}. A need was identified for a system that could accurately measure a patient's total body volume using a single piece of commodity hardware such as the Kinect.\\ 

The traditional method of measuring circumference of body parts such as arms or torso is limited by the accuracy of the physical device used to measure it, usually a tape measure. A bigger problem, however, arises when taking multiple readings. For the measurements to be useful the exact same point must be measured each time. A need was stated for the new toolkit to accurately identify the circumference of individual body parts, either from the original body scan or a new individual scan of the limb, and to enable the registration of and consistent guidance to the point of measurement.\\ 

The depth of subcutaneous fat in a person can also be measured to determine a patient's weight loss or variation in body size, and the depth of the fat is also an indicator of other serious conditions such as insulin resistance \cite{Goodpaster1997} and coronary heart disease \cite{Ducimetiere1986}. The problem here is measuring the depth of this subcutaneous fat consistently by placing the ultrasound scanner in precisely the same place, at the same angle, when taking multiple readings of the same body part over time. If this is not the case, errors can be introduced and may give a false indicator of loss or gain in subcutaneous fat. The PARSE team feel that the proposed toolkit can aid this, by recording the position of the first scan and providing direction to the user for subsequent scans. This guided positioning will be achieved using a mixture of the Kinect skeletal tracking and object recognition algorithms on the depth and image feeds from the Kinect scanner.\\ 